\documentclass[10pt,a4paper]{article}

%% Preamble
\usepackage[utf8]{inputenc}
\usepackage{geometry}           % See geometry.pdf to learn the layout options. There are lots.
\geometry{a4paper}              % ... or a4paper or a5paper or ... 
\usepackage[parfill]{parskip}   % Activate to begin paragraphs with an empty line rather than an indent
\usepackage{graphicx}
\usepackage{mathtools}
\usepackage{feynmp-auto}

\title{Particle Physics and its Feynman Representations}
\author{
    Paul Anthony Kiernan \\
    \texttt{pak79@cornell.edu}
}
\date{\today}


%% Begin document
\begin{document}
    \unitlength=1mm

\maketitle

\begin{abstract}
    This project represents the notes of a two-part project: an exploration of
    particle physics and its tools, and \LaTeX. I first explore the scattering
    processes of particles with well-defined momenta using the graphical
    technique popularized by Richard Feynman. The examination of these
    scattering processes are then illustrated using various \LaTeX  packages.
    \newline
    \newline
    Such meow! Yay!
\end{abstract}

\tableofcontents
\listoffigures

\newpage
\section{Feynman Diagrams}

    To calculate the probabilities for relativistic scattering processes we
    need to find out the Lorenz-invariant scattering amplitude which connects
    an initial state $|\Psi_{i}\rangle$ containing particles of well defined
    momenta to a final state $|\Psi_{f}\rangle$ other (often different)
    particles also with well-defined momenta.

    \subsection{Canonical Quantization Formulation}
        The probability amplitude for a transition of a quantum system from the
        initial state $|i\rangle$ to final state $|f\rangle$ is given by the
        matrix element
            \begin{equation}
                S_{fi} = \langle f | S | i \rangle
            \end{equation}
        where S is the scattering matrix.

        In the cannonical quantum field theory the scattering matrix, or
        S-matrix, is represented within the interaction picture by the
        perturbation series in the powers of the interaction Lagrangian,
            \begin{equation}
                S = \sum_{n=0}^{\infty} \frac{i^{n}}{n!} 
                    \int \prod_{j=1}^{n} d^4 x_j T  \prod_{j=1}^{n} L_v (x_j)
                    \equiv \sum_{n=0}^{\infty} S^{(n)}
            \end{equation}
        where $L_v$ is the interaction Lagrangian and $T$ signifies the time-
        ordered product of operators.

        A Feynman is a graphical representation of a term in the Wick's
        expansion of the time-ordered product in the $n$-th order term
        $S^{(n)}$ of the S-matrix.
            \begin{equation}
                T \prod_{j=1}^{n} L_v(j_j) =
                \sum_{\text{all possible contractions}} (\pm) N \prod_{j=1}^{n} L_v (x_j)
            \end{equation}
        where $N$ signifies the normal-product of the operators and $(\pm)$
        takes care of all the possible sign change when commuting the fermionic
        operators to bring them together for a contraction (a propogator).


\newpage
\section{Scattering Processes}
    \subsection{Compton Scattering}

        A photon of wavelength $\lambda$ comes in from the left, collides with
        a target at rest, and a new photon of wavelength $\lambda$ emerges at
        an angle $\theta$.

        \begin{figure}[h]
            \centering
            \begin{fmffile}{compton}
                \begin{fmfgraph*}(100,100)
                    \fmfleft{i1,i2}
                    \fmfright{o1,o2}

                    \fmflabel{$\gamma$}{i2}
                    \fmflabel{$e^-$}{i1}
                    \fmflabel{$\gamma$}{o1}
                    \fmflabel{$e^-$}{o2}

                    \fmf{photon}{i2,v2}
                    \fmf{fermion}{i1,v1,v2,o2}
                    \fmf{photon}{v1,o1}
                \end{fmfgraph*}
            \end{fmffile}

            \caption[Compton Scattering]{Feynman diagram for Compton scattering}
            \label{fig:compton-scattering}
        \end{figure}

\newpage
\section{Higgs Decays}
    \subsection{Gluon Decay}
        \subsubsection{Dual Gluon Decay}

            So many meows.

            \begin{figure}[h]
                \centering
                \begin{fmffile}{h_gg}
                    \begin{fmfgraph*}(100,100)
                        \fmfleft{i1}
                        \fmfright{d1,d2,o1,o2,d3,d4}
                        \fmflabel{$h$}{i1}
                        \fmflabel{$g$}{o1}
                        \fmflabel{$g$}{o2}
                        \fmf{dashes_arrow,tension=2}{i1,v1}
                        \fmf{quark,label=$t$,label.side=right}{v1,v2}
                        \fmf{quark,label=$t$}{v2,v3}
                        \fmf{quark,label=$t$}{v3,v1}
                        \fmf{gluon}{v2,o1}
                        \fmf{gluon}{v3,o2}
                        \fmfdot{v1,v2,v3}
                        \fmffreeze
                        \fmfshift{11down}{v2}
                        \fmfshift{11up}{v3}
                    \end{fmfgraph*}
                \end{fmffile}

                \caption[Gluon-gluon Decay]{Feynman diagram for Higgs Field Gluon-Gluon Decay}
                \label{fig:compton-scattering}
            \end{figure}
\end{document}

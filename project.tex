\documentclass[10pt,a4paper]{article}

%% Preamble

% Declare Package Imports
\usepackage[utf8]{inputenc}
\usepackage{geometry}           % See geometry.pdf to learn the layout options. There are lots.
\geometry{a4paper}              % ... or a4paper or a5paper or ... 
\usepackage[parfill]{parskip}   % Activate to begin paragraphs with an empty line rather than an indent
\usepackage{graphicx}
%\usepackage{epstopdf}
\usepackage{mathtools}
%\usepackage{mpgraphics}
%\usepackage{pst-pdf}
%\usepackage[final]{feynmp}
\usepackage{feynmp-auto}

%\DeclareGraphicsRule{.1}{mps}{*}{}
%\DeclareGraphicsRule{.tif}{png}{.png}{`convert #1 `dirname #1`/`basename #1 .tif`.png}

\title{Particle Physics and its Feynman Representations}
\author{
    Paul Anthony Kiernan \\
    \texttt{pak79@cornell.edu}
}
\date{\today}


%% Begin document
\begin{document}
    \unitlength=1mm

\maketitle
\tableofcontents
\listoffigures

\newpage
\section{Standard Model Interactions \\(Forces Meditated by Gauge Bosons)}
    \subsection{Fermion-Fermion Interactions}
        \subsubsection{Compton Scattering}

            A photon of wavelength $\lambda$ comes in from the left, collides
            with a target at rest, and a new photon of wavelength $\lambda$
            emerges at an angle $\theta$.

            \begin{figure}[h]
                \centering
                \begin{fmffile}{compton}
                    \begin{fmfgraph*}(100,100)
                        \fmfleft{i1,i2}
                        \fmfright{o1,o2}

                        \fmflabel{$\gamma$}{i2}
                        \fmflabel{$e^-$}{i1}
                        \fmflabel{$\gamma$}{o1}
                        \fmflabel{$e^-$}{o2}

                        \fmf{photon}{i2,v2}
                        \fmf{fermion}{i1,v1,v2,o2}
                        \fmf{photon}{v1,o1}
                    \end{fmfgraph*}
                \end{fmffile}

                \caption[Compton Scattering]{Feynman diagram for Compton scattering}
                \label{fig:compton-scattering}
            \end{figure}

\newpage
\section{Higgs Decays}
    \subsection{Gluon Decay}
        \subsubsection{Dual Gluon Decay}

            So many meows.

            \begin{figure}[h]
                \centering
                \begin{fmffile}{h_gg}
                    \begin{fmfgraph*}(100,100)
                        \fmfleft{i1}
                        \fmfright{d1,d2,o1,o2,d3,d4}
                        \fmflabel{$h$}{i1}
                        \fmflabel{$g$}{o1}
                        \fmflabel{$g$}{o2}
                        \fmf{dashes_arrow,tension=2}{i1,v1}
                        \fmf{quark,label=$t$,label.side=right}{v1,v2}
                        \fmf{quark,label=$t$}{v2,v3}
                        \fmf{quark,label=$t$}{v3,v1}
                        \fmf{gluon}{v2,o1}
                        \fmf{gluon}{v3,o2}
                        \fmfdot{v1,v2,v3}
                        \fmffreeze
                        \fmfshift{11down}{v2}
                        \fmfshift{11up}{v3}
                    \end{fmfgraph*}
                \end{fmffile}

                \caption[Gluon-gluon Decay]{Feynman diagram for Higgs Field Gluon-Gluon Decay}
                \label{fig:compton-scattering}
            \end{figure}
\end{document}
